\Conclusion % заключение к отчёту

В ходе работы был осуществлен обзор существующих алгоритмов локализации
человеческих лиц на изображении и определения пола и возраста человека:
основанных на геометрических особенностях объектов изображений и базирующихся на
текстурных шаблонах изображения, цветовых характеристиках. Также были
рассмотрены различные методологии и подходы применяемые для достижения цели
работы: использование нейронных сетей и метода опорных векторов. 

Существующие решения позволяют добиться относительно высоких показатели
точности, но в значительной степени зависят от области применения. Поэтому для
дальнейшей работы были выделены следующие этапы:

\begin{enumerate}
\item определение области применения;
\item осуществление подробного анализа существующих алгоритмов;
\item анализ ключевых точек и установление инвариантных особенностей
человеческого лица;
\item сравнение существующих алгоритмов, выделение их достоинств и недостатков; 
\item проведение экспериментов с целью определения подходящих алгоритмов;
\item измерение показателей точности и быстродействия алгоритмов;
\item синтез алгоритма обработки изображений для определения пола и возраста
человека.
\end{enumerate}


%%% Local Variables: 
%%% mode: latex
%%% TeX-master: "rpz"
%%% End: 
