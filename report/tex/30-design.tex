\chapter{Конструкторская часть}
\label{cha:design}


\section{Вступление}
Одной из областей компьютерного зрения ялвяется --- анализ человеческого лица по цифровому изображению.
Большинство существующих алгоритмов настолько ресурсоемки, что время их выполенения достаточно медленное для того, чтобы решать задачи в реальном режиме времени.
Однако с улучшением аппаратной базы, увеличением частоты процессора, данная проблема становится менне значимой (Piccardi and Jan, 2003).
На передний план выходит проблема, связанная с тем, что на данный момент не существует надежного алгоритма, который одинаково справлялся бы с большинством задач.

Например, комерческие разработки для идентификации личности, основанные на распознавании частей лица, имеют множество ограничений (Pentland, 2000). 

Такие области науки как распознавание образов по шаблонам и машинное обучение тесно связаны с областью компьютерного зрения. Алгоритмы распознавания и машинного обучения часто применяются в качестве составляющих в алгоримах компьютерного зрения. К примеру, существует множество алгоримов для выделения из видеопотока человеческих лиц и их анализа использхующих нейронные сети (Abdi et al., 1995; Golomb et al., 1990; Gray et al., 1995; Huang and Shimizu, 2006; Rowley et al., 1998; Tamura et al., 1996), Adaboost (Huang et al., 2004; Shakhnarovich et al., 2002; Sun et al., 2006; Viola and Jones, 2001; Wu et al., 2003), скрытые марковские модели (СММ) (Aleksic and Katsaggelos, 2006; Kohir and Desai, 1998; Yin et al., 2004), метод опорных векторов (SVM — support vector machines) (BenAbdelkader and Griffin, 2005; Castrillon-Satana et al., 2005; Moghaddam and Yang, 2000; Saatci and Towm, 2006; Sun et al., 2002; Yang et al., 2006).

В рамках компьютерного зрения анализ человеческого лица применяется для решения следующих задач:


\begin{itemize}
	\item обнаружение лица в видеопотоке и его отслеживание;
	\item идентификация личности;
	\item определение пола;
	\item определение возраста;
	\item определение эмоций;
	\item распознавание жестов;
	\item определение этнической принадлежности.
\end{itemize}

Несмотря на огромный прогресс достигнутый учеными за последние годы, решение для многих проблем так и не было предложено. Например, до сих пор не разработана система, которая бы позволила осуществлять обнаружение лиц и надежно идентифицировать по ним личность человека во всех возможных положениях лица. Фронтальное изображение лица в значительной степени отличется от изображения профиля, также возможны различные перспективные искажения, в том числе играют роль условия освещенности. Стоит отметить, что иногда возможны случаи, когда определить личность по изображению лица крайне трудно или вообще невозможно даже самим человеком.



В данной главе производится описание процесса распознавания образов человеческим зрением, также производится обзор подходов к цифровой обработке изображений, примение алгоритмов машинного обучения и алгоритмов распознавания образов.

\section{Человеческое зрение}
Для понимания проблем компьютерного зрения и автоматической обработки человеческих лиц полезно узнать процесс распознавания образов человеческим зрением.

В своем исследовании Tsao (2006) показал, что специальные клетки мозга чувствительны к воспринятию отдельных черт лица. В своих экспериментах авторы использовали лица, состоящие из картонных элементов. Изменяя положение девятнадцати составляющих, включая, положение глаз, носа, губ, размеры головы, ученые исследовали активность клеток головного мозга подопытных макак. В результате ученые выяснили, что одна клетка мозга восприимчива к изменению положения восьми составляющих. Также ученые выделили некоторые составляющие, изменение которых приводило к наибольшей активности отдельных секторов мозга.

\begin{figure}
\label{}
\end{figure}

Очевидно, что процесс распознавания образов с помощью человеческого зрения достаточно сложен и до конца не изучен. Сложность компьютерного зрения обуславливается анализом изображений, получаемых устройством захвата: разбиение входных изображений на отдельные части, объединение частей для получения смысловых элементов, и определение значения полученных элементов. Понятие процессов происходящих при человеческом зрении многим ученым позволило применить используемые в них правила и подходы для разработки алгоритмов компьютерного зрения.

\section{Компьютерное зрение}
Область компьютерного зрения весьма обширна. Применения находят как и промышленные приложения, так и приложения для домашнего использования. Однако, многие основные процессы и технологии являются одинаковыми при разработке приложений для различных сфер. В этой главе производится обзор этих подходов. Цифровая обработка изображений --- базис для дальнейшей обработки более высокого уровня, такой как распознавание образов. Также рассматриваются методы машинного обучения и распознавания образов, так как они получили широко применение в сфере компьютерного зрения.

\subsection{Цифровая обработка изображений}
Цифровая обработка изображений является базисной для всех систем компьютерного зрения. Существует несколько способов обработки изображений.

Цифровые камеры оснащаются либо КМОП (комплементарная логика на транзисторах металл-оксид-полупроводник), либо CCD-матрицами, которые преобразуют входной световой поток и преобразует его в разницу напряжений. Захват светового потока осуществляется с помощью светочувствительных датчиков. Количество датчиков зависит от типа камеры. Стандартные веб-камеры имеют размер матрицы 640*480. В настоящее время существуют камеры, оснащенные CCD-матрицами размером до 6 МегаПикселей.  

После преобразования светового потока в разницу напряжений происходит так называемое квантование. Это означает, что напряжение в определенном диапазоне отождествляется со строго определенным значением. Итогом процедуры квантования является изображение, представленное в цифровом виде, которое может в дальнейшем быть сохранено для последующей обработки.

Низкоуровневая обработка изображений может включать в себя фильтрацию в пространственной и частотной области. Основной задачей низкоуровневой обработки является улучшение качества изображения (Gonzalez and Woods, 2002, pp. 25-28). Также могут применяться процедуры морфологической обработки изображений и из сегментации на составляющие части.

Одной из задач фильтрации входных изображений является получение фильтрованного изображения, в котором интенсивности всех пикселей растянуты на более широком диапазоне. Применение подобной фильтрации позволяет снизить влияние таких факторов как ошибки в калибровке камеры, а также повысить контрастность изображения (Rowley et al., 1998). В рамках задачи анализа человеческого лица очень важно снизить влияние внешних условий на изменение изображения лица.

Преобразование гистограммы исходного изображения для усреднения интенсивностей является просто реализуемым. Для получения гистограммы применяют формулу:


%%% Local Variables:
%%% mode: latex
%%% TeX-master: "rpz"
%%% End:
