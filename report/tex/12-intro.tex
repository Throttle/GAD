\Introduction

Одним из наиболее важных направлений в развитии систем распознавания образов
является распознавание человеческих лиц. За последние десятилетия было создано
множество научных трудов, в которых ученые уделяли внимание конкретным
разработкам в области распознавания человеческих лиц, что свидетельствует о
возрастающей актуальности данной проблемы.

В связи с развитием компьютеризации и информатизации общества широкое
распространение получают автоматические системы идентификации. Повышенное
внимание к автоматическим системам идентификации личности обуславливается их
применимостью для решения ряда социальных и коммерческих задач. Идентификация
человека по изображению его лица применяется в системах безопасности,
видеонаблюдения, криминологии, а также в многих других сферах жизнедеятельности
человека. Основными требованиями к подсистеме идентификации являются скорость
работы и точность. 

В настоящее время создано значительно количество научных публикаций,
посвященных проблеме идентификации личности. Среди определяющих характеристик в
процессе идентификации рассматриваются: рост человека, комплекция, внешний
облика, походка, голос. Однако, ключевой характеристикой является --- лицо
человека. Предполагается, что анализируя изображение лица человека,
помимо утстановления конкретных биометрических показателей возможно определить
его пол и возраст. 

В течение последнего десятилетия с развитием современных систем идентификации
личности многие ученые, работающие в сфере машинного зрения, занимались
созданием эффективных алгоритмов определения пола и возраста людей по их
фотопортретам.

В результате были разработаны алгоритмы, которые позволяют автоматизировать
процесс установления пола и возраста человека. Множество классификаторов такие
как метод опорных векторов, нейронные сети, линейные и квадратичные
классификаторы, использовались для разделения людей по половому и возрастному
признаку. Однако до сих пор не существует метода, позволяющего со 100\%
точностью решить рассматриваемые задачи. 

Системы определения пола и возраста человека могут быть разбиты на две основные
категории в зависимости от представления входного изображения лица и его
обработки. Схемы, основанные на обработке мета-данных изображения, используют
текстовое описание изображения для его представления и обработки. В то время как
другой тип в других схемах изображение представляется и обрабатывается как набор
геометрических объектов и текстур.

Большинство существующих систем идентификации являются
шаблонно-оринетированными, то есть для определения личности используются
определенные характеристические черты. Поэтому одной из важнейших составляющих
данных систем является алгоритм выделения характеристических черт, к которому
предъявляются требования по работе в реальном режиме времени и точности. Не
менее важной является задача определения того, какие характеристические черты
следует использовать в ходе анализа.


Определение пола и возраста человека играет важнейшую роль при идентификации
личности. Задачей идентификации является установление тождественности
неизвестного объекта известному на основании совпадения признаков. Обычно в
системах имеется база данных, содержащая перечень известных объектов. В таких
системах предварительный анализ изображения для определения пола может быть
применен для сокращения времени стадии поиска путем выбора соответствующей базы
данных. После определения пола могут быть применена классификация по другим
характеристическим особенностям, например, возрасту, что при корректной
организации базы данных может привести к увеличению производительности в
несколько раз.

Также для маркетинговых и статистических исследований требуется информация об
определенных группах людей, которые выделяются с помощью автоматических средств
распознавания. В силу некритичности погрешностей в некоторых случаех снижается
требование повышенной точности алгоритмов распознавания.

Таким образом можно констатировать, что спектр применения алгоритмов определения
характеристических особенностей человека, в частности пола и возраста, широк. В
конкретной области применения на данные алгоритмы накладываются определенные
требования по точности, скорости выполнения. Одной из существующих проблем
является реализация подобных алгоритмов для работы с искаженными,
перспективными, а также имеющими низкое разрешение динамическими изображениями,
например, видеопотоком. В силу этого тема настоящей работы является актуальной.

{\bf Целью настоящей работы} является разработка алгоритма определения пола и
возраста человека по видеопотоку веб-камеры.

Цель достигается в несколько этапов:
\begin{enumerate}
\item анализ предметной области:
	\begin{enumerate}
		\item анализ возможных сфер применения;
		\item анализ существующих алгоритмов; выделение достоинств и
недостатков;
		\item формулировка требований;
	\end{enumerate}
\item программная реализация автоматического классификатора объектов видеопотока
по половому и возрастному признаку.
\item исследование для выявления области применения.
\end{enumerate}

{\bf Объектом изучения} является автоматическая обработка изображений. {\bf
Предметом изучения} --- алгоритмы определения пола и возраста человека по
изображениям лица.

%Для успешного изучения описанной проблемы в настоящей работе
%рассматриваются следующие методы научного познания:

%\begin{enumerate}
%\item эксперимент;
%\item измерение;
%\item сравнение;
%\item анализ и синтез.
%\end{enumerate}

%Также в работе описывается новая парадигма в методологии науки --- синергетика.

% По результатам настоящей работы готовится статья для публикации в научном
% электронном журнале <<Наука и образование>>\footnote{Рецензируемый электронный
% журнал <<Наука и образование: электронное научное издание. Инженерное
% образование>> (ISSN 1994-0408, № Гос. регистрации 0420800025, ЭЛ № ФС
% 77-30569)}.

