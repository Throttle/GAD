\chapter{Аналитическая часть}
\label{cha:analysis}
%
% % В начале раздела  можно напомнить его цель
%
В данном разделе производится обзор предметной области, освещаются проблемы,
возникающие в процессе распознавания образов, связанные с особенностями
обрабатываемых изображений. Также приводятся исторические сведения
подтолкнувшие человечество к созданию автоматических систем распознования
личностей по фотопортретам.

Основным элементом данного раздела является обзор
существующих алгоритмов обработки фотопортретов для определения пола и возраста
изображенных людей.

\section{История идентификации личности}
Распознавание образов происходит в повседневной жизни человека. Человек почти
мгновенно узнаем знакомого человека по внешнему виду в толпе или по голосу в
телефонном разговоре.


Существуют бумажные и биометрические идентификаторы личности. Бумажные
идентификаторы --- это паспорт, удостоверение личности, водительское
удостоверение. Существуют также пароли, персональные идентификационные номера
ПИН-коды. Но паспорт, удостоверение личности, водительское удостоверение можно
потерять, сравнительно легко подделать. Пароль, ПИН-код можно забыть или
перепутать.

Биометрические идентификаторы основаны на физиологических или поведенческих
характеристиках, сугубо индивидуальных для каждого человека. К ним относятся
внешность, походка, почерк, отпечатки пальцев, радужная оболочка глаз, форма
лица. Эти уникальные качества каждого человека трудно подделать,
невозможно забыть или потерять. В этом и состоят их очевидные преимущества перед
бумажными идентификаторами.

Основателем использования биометрической идентификации стал французский
криминалист Альфонс
Бертильон (1853 – 1914 гг.)\footnote{Свободная
энциклопедия Википедия: \url{http://en.wikipedia.org/wiki/Alphonse_Bertillon}}.
Он первым предложил использовать выводы антропологов о том, что геометрические
размеры частей тела у разных людей никогда не совпадают полностью. Начиная с
1883 года, он измерял преступников и заносил данные о них в картотеку. Этот
метод получил название бертильонажа. Впоследствии Бертильон усовершенствовал
свой метод и фиксировал информацию о преступниках в виде фотопортретов: анфас и
профиль.

Сегодня применяются, в основном, три метода биометрической идентификации:
распознание по отпечаткам пальцев, по радужной оболочке глаз и по форме лица.
Появились новые биологические средства и методы распознавания личности,
например, по найденным образцам ДНК. Однако они требуют значительного времени и
не относятся к оперативным средствам распознавания.

В этих условиях особую важность приобретают оперативные автоматизированные
методы распознавания личности, в которых время распознавания играет основную
роль: в людских потоках на эскалаторах метро, в очередях регистрации
авиапассажиров. Разумеется, эти методы обязательно дополняются методами
обнаружения оружия и взрывчатки.

\section{Исследование рынка систем определения пола и возраста человека}

Множество существующих комерческих систем в сфере компьютерного зрения в
качестве одной из своих функциональных возможностей позволяют определять пол и
возраст человека по последовательности кадров видеопотока. В данной главе
рассматриваются программные продукты, с точки зрения определения функциональных
возможностей и результатов подсистем определения пола и возраста.

Из коммерческих решений стоит выделить систему распознавания эмоций
<<FaceReader>> голландской компании <<Noldus Information Technology>>
\footnote{Noldus Information Technology: \url{http://www.noldus.com/office/ru}}.
Данный продукт является наиболее совершенным на рынке распознавания эмоций и в
то же время позволяет решать многие другие задачи.

Достоинствами системы являются:
\begin{enumerate}
 \item высокий процент точности определения пола ($89\%$) и возраста человека;
 \item возможность работы без предварительного обучения и настройки;
 \item использование нескольких алгоритмов машинного зрения: Active Templates и
Appearance Active Models;
 \item наклон лица может быть любым в плоскости изображения, его система
обнаружит;
 \item работа с загружаемыми видео в форматах с кодеками MPEG1, MPEG2,
XviD, DivX4, DivX5, DivX6, DV-AVI и uncompressed AVI;
 \item возможность просмотра гистограмм, диаграмм, используемых при обработке
изображения;
 \item работа в реальном режиме времени.
\end{enumerate}

К недостаткам программы можно отнести:
\begin{enumerate}
 \item невозможность распознавания детей до 5ти лет;
 \item если человек в очках, то распознавание неточное, либо
классификация не ведется;
 \item люди с разным цветом кожи по-разному воспринимаются системой, программа
не до конца адаптирована;
 \item повернутое лицо не детектируется.
\end{enumerate}

Американская компания <<L1>> разработала продукт <<Face-It>>\footnote{L1
Identity Solutions: \url{http://www.l1id.com/}}. Очевидными достоинствами
программного комплекса являются:
\begin{enumerate}
 \item определение пола и возраста независимо от цвета кожи;
 \item возможность детектирования лиц даже, если человек в очках;
 \item работа в реальном режиме времени.
\end{enumerate}

Из недостатков можно выделить:
\begin{enumerate}
 \item разрешение входного изображения ограничено размером $320\times240$
пикселей. 
\end{enumerate}

Немецкая компания <<Cognitec Systems>> предоставляет систему
<<FaceVACS-PortraitAcquisition>>\footnote{Cognitec Systems:
\url{cognitec-ag.de}}, позволяющую производить детектирование лиц в видопотоке
и в том числе осуществлять гендерную классификацию. Из заявленных
производителем достоинств можно выделить подавление перспективных искажений, а
также работу с искаженным и зашумленным входным потоком. Весомым недостатком
является слабая документированность системы.

В результате произведенного обзора существующих систем можно выделить
следующие общие характеристики:
\begin{enumerate}
 \item большинство компаний за основу алгоритмов распознавания лиц на
изображении, а также определения возраста и пола используют известные алгоритмы,
однако с корпоративными доработками;
 \item во всех продуктах не решена проблема распознавания при повороте лица вне
плоскости изображения.
\end{enumerate}

\section{Обнаружение и локализация лица на изображении}
При обнаружении и локализации лиц на изображениях видеопотока приходится
сталкиваться со следующими трудностями:

\begin{enumerate}
  \item варьирующийся внешний вид лиц (размер, цвет кожи);
  \item ориентация в пространстве приводит к значительным искажениям
изображения;
  \item мимика;
  \item наличие специальных особенностей (очки, усы, борода);
  \item экранирование лиц другими объектами сцены;
  \item особенности обрабатываемого изображения (освещенность, калибровка
камеры, размер изображения, качество изображения).
\end{enumerate}

В задаче локализации лиц на изображении обычно применяют алгоритмы, которые
можно разбить на две категории: основанные на методах, пытающиеся формализовать
алгоритм с точки зрения работы человеческого мозга, и на методах, используемых
в задачах распознавания.

\section{Обзор существующих алгоритмов}

Существующие алгоритмы обнаружения лиц можно разбить на две категории. К
первой категории относятся методы, отталкивающиеся от опыта человека в
распознавании лиц и делающие попытку формализовать и алгоритмизовать этот опыт,
построив на его основе автоматическую систему распознавания. Вторая категория
опирается на инструментарий распознавания образов, рассматривая задачу
обнаружения лица, как частный случай задачи распознавания. 

В дальнейшем согласно \cite{Veznevec_Degtyareva} алгоритмы первой категории
будем относить к эмпирическому распознаванию, второй --- к моделированию
изображения лица. Авторы \cite{Veznevec_Degtyareva} в своей работе приводят
наиболее полную классификацию данных алгоритмов.

\subsection{Эмпирическое распознавание}

Авторы алгоритмов данного типа стремились использовать принципы, которыми
руководствуется человеческий мозг при решении задачи распознавания образов.
Выделяют два направления методов данной категории: основанные на знаниях и
основанные на особенностях.

Распознавание, основанное на знаниях, заключается в построении набора правил, с
помощью которого будет найден фрагмент изображения, являющийся
человеческим лицом. Например, одними из правил могут быть правила симметрии
лица (глаза, брови), пигментации губ и так далее, расположения определенных
черт лица. Алгоритм заключается в проверке всех фрагментов изображения на
выполенение всех правил.

Улучшенной модификацией данного класса алгоритмов являются
шаблонно-ориентиованные алгоритмы. Пользователем заранее создаются наборы, так
называемых шаблонов, с которыми производится сранение фрагментов изображения.
Фактически одного сравнения достаточно для определения является ли данный
фрагмент человеческим лицом или нет.

Другой класс алгоритмов распознавания основывается на выделении особенностей,
то есть инвариантных свойств изображений человеческого лица, что позволяет
отделить его от фона и других объектов сцены.

В качестве инвариантных свойств выделяют:
\begin{enumerate}
 \item границы (ребра) объектов;
  \item цвет (кожа);
  \item форма черт лица (форма губ, овала лица и так далее);
  \item комбинированные свойства.
\end{enumerate}

Следующим шагом алгоритмов данного класса является определение областей со
связными особенностями и их проверка на соответствие изображению человеческого
лица. 

Принцип заложенный в данном подходе получил популярность у многих ученых:
\begin{enumerate}
 \item K. Sobottka в своей диссертации \cite{K_Sobottka} предложил выделять
инвариантные свойства на основе их яркости в предположении, что такие черты
лица как губы, глаза, брови, имеют низкую яркость в отличие от окружающих
частей лица;
  \item F. Smeraldi реализовал алгоритм \cite{F_Smeraldi}, использующий
положение глаз для локализации человеческого лица. 
\end{enumerate}


\subsection{Моделирование изображения лица}
Основной задачей алгоритмов данной категории является построение математической
модели человеческого лица, применяя методы математической статистики и
машинного обучения.

В общем случае фрагменту изображения ставится в соответствие некоторым образом
вычисленый вектор признаков, который используется для классификации изображения
на два класса --- лицо, либо не лицо.

В алгоритмах данного типа используются:
\begin{enumerate}
  \item моделирование класса изображений лиц с помощью Метода Главных Компонент;
  \item моделирование класса изображений лиц с помощью Факторного анализа;
  \item Линейный Дискриминантный Анализ (Linear Discriminant Analysis, LDA);
  \item нейронные Сети (Neural Networks, NN);
  \item скрытые Марковские Модели (Hidden Markov Models, HMM);
  \item Sparse Network of Windows (SNoW);
  \item Active Appearance Models (AAM).
\end{enumerate}

Для классификации фрагментов в основном используют метод опорных векторов.

Входными данными для перечисленных алгоритмов является поток цифровых
изображений.

Под цифровым изображением при реализации данных алгоритмов понимается случайный
двумерный дискретный сигнал, наблюдаемый системой. Последовательность
наблюдений, то есть вектор наблюдений может извлекаться из изображения
различными способами. В силу этого описательные способности полученных моделей
могут различаться. Наиболее популярным является вариант сканирования изображения
прямоугольным окном. Для уменьшения вероятности потери данных на границах
блоков, сканирование изображений осуществляется таким образом, что соседние
блоки пикселей перекрываются друг другом. Значение перекрытия задается в
качестве параметра алгоритма.

Для снижения вычислительной сложности и уменьшения пространства признаков,
каждый извлеченный блок пикселей подвергается некоторому преобразованию, в
результате которого получается некоторый набор числовых данных, который и
является вектором наблюдений.

В каждом алгоритме используется конкретный математический аппарат, позволяющий
справиться с поставленной задачей локализации человеческого лица.

//TODO: более полное описание существующих алгоритмов. VIOLA-JHONES.

\subsection{Сравнительный анализ}

К недостаткам эмпирического подхода можно отнести высокую чувствительность
алгоритма к изменчивости объекта распознавания, зависимость от условий съемки и
освещения.

Несомненным достоинством является то, что применение эмпирических правил
позволяет снизить сложность проверок при обработке изображения для локализации
лиц.

В целом основной трудностью при реализации алгоритмов эмпирического типа
является сложность построения эмпирических правил: поскольку чересчур жесткие
рамки правил приведут к тому, что в ряде случаев лица не будут обнаружены, и
напротив, слишком общие правила приведут к большому количеству случаев ложного
обнаружения. 

Основанные на классификаторах алгоритмы второй категории чувствительны к
изменениям ориентации и масштаба лица, так как большинство из классификаторов
не являются инвариантными к повороту лица и изменению его размера. Это приводит
к необходимости в предварительной обработке входного изображения:
масштабирование, поворот, перспективные преобразования; либо составлению более
полного тренировочного набора. В целом алгоритмы второй категории обладают
более высокой вычислительной сложностью, что затрудняет использование некоторых
из них в системах реального времени.


К общим трудностям многих алгоритмов обеих категорий можно отнести
восприимчивость к повороту лиц вне плоскости изображения. Правила алгоритмов
эмпирического типа в данном случае становятся совершенно непригодными.

Несмотря на наличие общих недостатков в алгоритмах обеих категории, существуют
методы позволяющие снизить влияние негативных факторов на результат
обнаружения. Например, проблему разницу в освещении, масштабирования решают
предварительной обработкой входного изображения.

В итоге можно сделать вывод, алгоритмы, использующие эмпирический подход,
просты в реализации и их целесообразно использовать для задач реального времени
с некритичными требованиями к точности обнаружения. Алгоритмы, основанные на
моделировании изображения лица, сложны в реализации, не всегда могут быть
применимы в задачах реального времени, однако позволят достичь более
высокой точностью обнаружения.

\section{Определение пола и возраста человека по изображению лица}
В своих работах многие ученые освещали способы решения проблемы, связанной с
обработкой изображений для определения пола и возраста человека. Авторы
предлагют различные методы такие как: метод опорных векторов, методы, основанные
на геометричеких особенностях, графовых моделях и методы на основе нейронных
сетей. В данной главе производится обзор различных подходов к решаемой проблеме:
подходе основанном на выделении геометрических особенностей и подходе на основе
шаблонов.


\subsection{Геометрический анализ}
Представление лиц  с точки зрения геометрических особенностей является самым
распространенным методом решения задач обработки лиц. Обычно выделяются
геометрические особенности инвариантные к  масштабированию, перспективным
искажениям и поворотам такие как расстояния, углы и отношения между частями
лица. Эти особенности представляют человеческое лицо и обеспечивают входные
данные для обучения классификатора, который осуществляет окончательную
классификацию.

Bruneli и Poggio \cite{Bruneli_Poggio} использовали данный подход в своих
работах для определения пола человека. Их алгоритм предполагает извлечение из
изображения лица человека шестнадцати геометрических особенностей. Впоследствии
эти данные используются для обучения двух различных классификаторов, отдельно
для мужчин и женщин. Описанный в \cite{Bruneli_Poggio} алгоритм позволяет
добиться 90\% точности распознавания для изображений из обучающего набора и 79\%
для новых изображений.

Первым этапом в любом алгоритме, основанном на геометрических параметрах объкта
распознавания, является извлечение характеристических особенностей из входного
изображения. В своей работе \cite{Yang_Huang} Yang и Huang описывают метод
основанный на иерархическом обучении для определения лиц людей на изображениях,
представляющих сложные динамические сцены с множеством других объектов.
Shackleton и Welsh \cite{Shackleton_Welsh} разработали алгоритм, позволяющий с
высокой точностью выделять глаза на фронтальном изображении лица человека. Wu и
Yokoyama \cite{Wu_Yokoyama} предоставляют основанный на информации о цвете
алгоритм, с помощью которого можно выделять различные геометрические черты лица
человека. В работе Eveno, Caplier и Coulon \cite{Eveno_Caplier_Coulon}
описывается алгоритм, выделения формы губ. Данный алгоритм основан на цветовой
информации и требует обучения с помощью входного множества цветных изображений
губ людей.


\subsection{Классификация с использованием специфических особенностей объекта}

В настоящее время создано множество приложений, базирующихся на выделении из
изображений лица таких специфических особенностей, как усы, борода и другие.
Применение подобной методики позволяет быстро отсеять точно неподходящие
варианты. Например, солжно представить женщину с бородой.

Lapedriza, Masip и Vitria \cite{Lapedriza_Masip_Vitria} предложили алгоритм,
основанный на низходящей обработке изображения для выделения специфических
особенностей. Процесс обучения в данном алгоритме отличается от рассмотренных
ранее тем, что обучаются лишь те участки изображения, в которых могут
присутствовать интересующие нас объекты. С использованием данного алгоритма
удалось добиться показателя точности 94\% для обучающей выборки из 90
изображений лиц (40 мужских, 40 женских).


\subsection{Нейронные сети}
Многие ученые для решения рассматриваемой задачи используют нейронные сети.
Colomb \cite{Colomb} предложил использовать нейронную сеть с обратной связью
размером $900\times40\times900$. В описанном алгоритме обрабатываемые
изображения преобразуются к размеру $30\times30$, также производится поворот
изображения так, чтобы на всех изображениях обучающей выборки позиции глаз и губ
были идентичны. На выходе сеть выдает $1$ --- если детектирован мужчина и $0$
--- если определена женщина. Средняя ошибка алгоритма \cite{Colomb} составляет
$8,1\%$.  

\subsection{Метод опорных векторов}
Метод опорных векторов (SVM, Support Vector Machine) является алгоритмом вида
<<обучения с учителем>> и используется для задач классификации и регрессионного
анализа. Основная идея метода опорных векторов --- перевод исходных векторов в
пространство более высокой размерности и поиск разделяющей гиперплоскости с
максимальным зазором в этом пространстве. 

Babac в своей работе \cite{Babac} провел исследование и установил, что
классификатор SVM является наиболее точным для изображений с низким
разрешением.


\subsection{Выбор алгоритма}
Авторы \cite{Veznevec_Degtyareva} предлагают осуществлять выбор алгоритма
обнаружения и локализации лиц на изображениях в соответствии со схемой
представленной в таблице \ref{tab:chose_algo}.

\begin{table}[ht]
  \caption{Выбор алгоритма обнаружения и локализации лиц на изображениях}
  \begin{tabular}{|p{0.25\textwidth}|p{0.70\textwidth}|}
  \hline
  Параметр & Значение \\ 
  \hline
  Предполагаемое разнообразие лиц & Ограниченный набор людей, ограничения на
возможный тип лица, отсутствие ограничений. \\ 
  \hline
  Ориентация лиц на изображении & Наклон под известным
углом, в определенных границах вблизи известного угла наклона, любая
ориентация. \\
  \hline
  Цветовая палитра & Цветное или черно-белое избражение. \\
  \hline
  Количество лиц & Фиксировано, неизветно. \\
  \hline
  Фон & Фиксированный, контрастный однотонный, слабоконтрастный зашумленный,
неизвестный. \\
  \hline
  Условия освещения & Фиксированные известные, приблизительно известные, любые.
  \\
  \hline
  \end{tabular}
  \label{tab:chose_algo}
\end{table}

Помимо этого авторы акцентируют внимание на таких параметрах как качество
обрабатываемых изображений, а также точности желаемого результата. В некоторых
задачах важно не пропустить ни одного лица, но допускаются случаи ложного
обнаружения. В других случаях возможны более строгие требования именно к
качеству обнаружения, для этого необходимо минимизировать количество случаев
ложного обнаружения.

Стоит отметить, что предложенная схема также подходит для задачи более высокого
уровня, то есть определения пола и возраста человека по изображению лица.

Из рассмотренных примеров видно, что некоторые характеристики объектов сцены
с одной стороны позволяют более эффективно устанавливать пол человека, а с
другой затруняют локализацию лица на изображении, например, к таким
характеристикам можно отнести усы и бороду.

\section{Выводы}
Многие ученые посвятили свои труды решению проблемы обработки изображений с
целью определения пола и возраста человека. Было установлено, что разработанные
на данный момент решения хотя и предоставляют относительно высокие показатели
точности, должны быть соотнесены с конкретной сферой применения для решения
конкретных задач.


%%% Local Variables: 
%%% mode: latex
%%% TeX-master: "rpz"
%%% End: 
